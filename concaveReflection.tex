%
% Copyright � 2012 Peeter Joot.  All Rights Reserved.
% Licenced as described in the file LICENSE under the root directory of this GIT repository.
%
% pick one:
%\input{../assignment.tex}
%\input{../blogpost.tex}
%\renewcommand{\basename}{concaveReflection}
%\renewcommand{\dirname}{notes/FIXMEwheretodirname/}
%%\newcommand{\dateintitle}{}
%%\newcommand{\keywords}{}
%\input{../peeter_prologue_print2.tex}
%\beginArtNoToc
%\generatetitle{FIXME put title here}

\makeproblem{Solve the geometry of a concave spherical mirror}{concaveReflection:pr:1}{
After solving, apply the paraxial approximation to find the ABCD matrix result. }

\makeanswer{concaveReflection:pr:1}{

Our system is illustrated in \cref{fig:concaveReflection:concaveReflectionFig1}.

\imageFigure{../figures/phy485-optics/concaveReflectionFig1}{Concave spherical reflector}{fig:concaveReflection:concaveReflectionFig1}{0.4}

From the figure, employing the law of sines, we have

\begin{equation}\label{eqn:concaveReflection:10}
t \sin\theta = r \sin\alpha = (r - s') \sin\theta_2,
\end{equation}

or
\begin{dmath}\label{eqn:concaveReflection:30}
\frac{r - s'}{\sqrt{s^2 + y^2}} \sin\theta_2
= \sin( \gamma + \theta_1)
= \sin \gamma \cos \theta_1
+ \cos \gamma \sin \theta_1,
\end{dmath}

but since
\begin{equation}\label{eqn:concaveReflection:50}
\gamma = \Atan \frac{y}{s},
\end{equation}

and

\begin{subequations}
\begin{equation}\label{eqn:concaveReflection:70}
\cos\Atan x = \inv{\sqrt{1 + x^2}}
\end{equation}
\begin{equation}\label{eqn:concaveReflection:90}
\sin\Atan x = \frac{x}{\sqrt{1 + x^2}},
\end{equation}
\end{subequations}

we have
\begin{dmath}\label{eqn:concaveReflection:110}
\frac{r - s'}{\sqrt{s^2 + y^2}} \sin\theta_2
=
\frac{y/s}{\sqrt{1 + (y/s)^2}} \cos \theta_1
+\frac{1}{\sqrt{1 + (y/s)^2}} \sin \theta_1
=
\frac{y}{\sqrt{s^2 + y^2}} \cos \theta_1
+\frac{s}{\sqrt{s^2 + y^2}} \sin \theta_1,
\end{dmath}

or
\boxedEquation{eqn:concaveReflection:130}{
(r - s') \sin\theta_2 = y \cos \theta_1 + s \sin \theta_1.
}

This is the exact result desired.  Application of the paraxial approximation gives us

\begin{equation}\label{eqn:concaveReflection:150}
(r - s') \theta_2 \sim y + s \theta_1.
\end{equation}

With

\begin{subequations}
\begin{equation}\label{eqn:concaveReflection:170}
\theta_2 \sim \frac{y'}{s'}
\end{equation}
\begin{equation}\label{eqn:concaveReflection:190}
\theta_1 \sim \frac{\Delta y}{r + s}
\end{equation}
\end{subequations}

we have
\begin{equation}\label{eqn:concaveReflection:210}
s \theta_1 \sim -r \theta_1 + \Delta y
\end{equation}
\begin{equation}\label{eqn:concaveReflection:230}
s' \theta_2 \sim y'
\end{equation}

\begin{equation}\label{eqn:concaveReflection:250}
r \theta_2 - y' \sim y - r \theta_1 + \Delta y,
\end{equation}

or
\begin{equation}\label{eqn:concaveReflection:270}
\theta_2 \sim \frac{2}{r} y' - \theta_1
\end{equation}

We need to fix the sign conventions for the ABCD matrices, so write

\begin{subequations}
\begin{equation}\label{eqn:concaveReflection:290}
\alpha' = -\theta_2
\end{equation}
\begin{equation}\label{eqn:concaveReflection:310}
R = -r
\end{equation}
\begin{equation}\label{eqn:concaveReflection:330}
\alpha = \theta_1
\end{equation}
\end{subequations}

A final substitution into \eqnref{eqn:concaveReflection:270} gives us
\begin{equation}\label{eqn:concaveReflection:270b}
\alpha' \sim \frac{2}{R} y' + \alpha
\end{equation}

or in matrix form
\begin{equation}\label{eqn:concaveReflection:350}
\begin{bmatrix}
y' \\
\alpha'
\end{bmatrix}
=
\begin{bmatrix}
1 & 0 \\
\frac{2}{R} & 1
\end{bmatrix}
\begin{bmatrix}
y \\
\alpha
\end{bmatrix}.
\end{equation}

This is the ABCD matrix we were given in class.
} % makeanswer

%\EndNoBibArticle
