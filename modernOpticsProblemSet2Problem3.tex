%
% Copyright � 2013 Peeter Joot.  All Rights Reserved.
% Licenced as described in the file LICENSE under the root directory of this GIT repository.
%
\makeoproblem{Solar interference}
{modernOptics:problemSet2:3}
{2012 Problem Set 2, Problem 3}
{
Let's consider the prospects for interference fringes using direct sunlight.

\makesubproblem{
Consider the sun to be a disc subtending a 0.5 degree diameter. Using the van Cittert-Zernike theorem, find the mutual coherence function on earth from sunlight. How close would two pinholes need to be to see a 50\% visibility interference pattern behind them? For this part, make the (wrong) assumption that the sun is a quasimonochromatic source centered at \(\lambda =\)500 nm.
}{modernOptics:problemSet2:3a}
\makesubproblem{
Another difficulty with the sun (when using it as a source for interferometry) is that it is spectrally broadband. As with any blackbody, a typical spectral width is \(\Delta \omega = \kB T/\Hbar\), where \(T \approx 5000\)\,K for the sun. {\em Estimate} the effect of finite coherence time on fringe visibility, and make a qualitative sketch of the fringe pattern you would expect to observe.
}{modernOptics:problemSet2:3b}
\makesubproblem{
In what situations is the spectral width a more severe problem for visibility than the spatial coherence?
}{modernOptics:problemSet2:3c}
} % makeoproblem

\makeanswer{modernOptics:problemSet2:3}{
\paragraph{Part \ref{modernOptics:problemSet2:3a}.  }

From the class notes (page 7, 11, 12) we have for the mutual coherence

\begin{equation}\label{eqn:modernOptics:ProblemSet2:P3:10}
\Gamma_{12}
=
e^{ i \Bk_{\mathrm{av}} \cdot \Delta \Br }
\inv{ \lambda^2 \overbar{R_1} \overbar{R_2} }
\iint d^2 r_s e^{-i \Bk_s \cdot \Delta \Br } I( \Bk_s )
\end{equation}

where

\begin{subequations}
\begin{equation}\label{eqn:modernOptics:ProblemSet2:P3:30}
\Bk_s = k \frac{\Br_s}{r_{\mathrm{av}}}
\end{equation}
\begin{equation}\label{eqn:modernOptics:ProblemSet2:P3:50}
\Bk_{\mathrm{av}} = k \rcap_{\mathrm{av}}
\end{equation}
\begin{equation}\label{eqn:modernOptics:ProblemSet2:P3:70}
\Delta \Br = \Br_1 - \Br_2
\end{equation}
\end{subequations}

Let's write for the disk radius \(R\), distance from the disk \(D\), separation of the observation points \(d\).  We'll place the observation points in the plane of the disk, symmetrically separated around the normal to the disk from the center
setup our coordinates as in \cref{fig:modernOpticsProblemSet2Problem3:modernOpticsProblemSet2Problem3Fig1}

\begin{subequations}
\begin{equation}\label{eqn:modernOptics:ProblemSet2:P3:90}
\Br_{\mathrm{av}} = D \zcap
\end{equation}
\begin{equation}\label{eqn:modernOptics:ProblemSet2:P3:110}
\Br_1 = \Br_{\mathrm{av}} + \frac{d}{2} \xcap
\end{equation}
\begin{equation}\label{eqn:modernOptics:ProblemSet2:P3:130}
\Br_2 = \Br_{\mathrm{av}} - \frac{d}{2} \xcap
\end{equation}
\begin{equation}\label{eqn:modernOptics:ProblemSet2:P3:150}
\Delta \Br = (d) \xcap
\end{equation}
\end{subequations}

\imageFigure{../../figures/phy485/modernOpticsProblemSet2Problem3Fig1}{Geometry for solar disk interference problem}{fig:modernOpticsProblemSet2Problem3:modernOpticsProblemSet2Problem3Fig1}{0.4}

We have

\begin{equation}\label{eqn:modernOptics:ProblemSet2:P3:170}
\rcap_{\mathrm{av}} \cdot \Delta \Br
= \zcap \cdot (d) \xcap
= 0,
\end{equation}

so our mutual coherence is reduced to

\begin{equation}\label{eqn:modernOptics:ProblemSet2:P3:190}
\Gamma_{12}
=
\cancel{ e^{ i \Bk_{\mathrm{av}} \cdot \Delta \Br} }
\inv{ \lambda^2 D^2 }
\iint d^2 r_s e^{-i k \frac{\Br_s}{r_{\mathrm{av}}} \cdot \xcap d } I( \Bk_s ).
\end{equation}

Using an approximation of constant intensity \(I(\Bk_s) = I_0\) over the disk, and employing radial coordinates

\begin{equation}\label{eqn:modernOptics:ProblemSet2:P3:210}
\Br_s = \rho (\cos\theta, \sin\theta)
\end{equation}

we have

\begin{equation}\label{eqn:modernOptics:ProblemSet2:P3:230}
\Gamma_{12}
=
\frac{I_0}{ \lambda^2 D^2 }
\int_0^{2 \pi} d\theta
\int_0^R \rho d\rho
e^{-i k \frac{d}{D} \rho \cos\theta }.
\end{equation}

With a substitution \(a = -k d/D\), and some supplication to Mathematica (\nbref{modernOpticsProblemSet2work.cdf}), we find for the integral

\begin{equation}\label{eqn:modernOptics:ProblemSet2:P3:250}
\int_0^{2 \pi} d\theta
\int_0^R \rho d\rho
e^{i a \rho \cos\theta }
=
\frac{2 \pi R J_1(a R)}{a},
\end{equation}

so that the mutual coherence is

\begin{equation}\label{eqn:modernOptics:ProblemSet2:P3:270}
\Gamma_{12}
=
\frac{I_0}{ \lambda^2 D^2 }
\frac{2 \pi R J_1(k d R/D)}{ k d /D }
=
\frac{I_0 2 \pi R^2 }{ \lambda^2 D^2 }
\frac{J_1(k d R/D)}{ k d R /D }.
\end{equation}

We note that from the point(s) of observation, the observed angle of the disk is

\begin{equation}\label{eqn:modernOptics:ProblemSet2:P3:290}
\theta_s \sim \frac{2 R}{D},
\end{equation}

and we also note that
\begin{equation}\label{eqn:modernOptics:ProblemSet2:P3:310}
k = \frac{2 \pi}{\lambda},
\end{equation}

so our Bessel argument can be rewritten as

\begin{equation}\label{eqn:modernOptics:ProblemSet2:P3:330}
\frac{k d R}{D} = \frac{2 \pi}{\lambda} d \frac{\theta_s}{2} = \frac{\pi d \theta_s}{\lambda}
\end{equation}

so that our mutual coherence in terms of desired variables is

\boxedEquation{eqn:modernOptics:ProblemSet2:P3:370}{
\Gamma_{12}
=
I_0 \frac{\pi \theta_s^2}{ 2 \lambda^2}
\frac{J_1(\pi \theta_s d/\lambda)}
{\pi \theta_s d/\lambda}
}

with

% FIXME: this is a fudge but seems right.  How would I_1 be calculated directly?
\begin{equation}\label{eqn:modernOptics:ProblemSet2:P3:390}
I_1 = I_2 = \evalbar{\Gamma_{12}}{d = 0} = I_0 \frac{\pi \theta_s^2}{ 2 \lambda^2} \inv{2}
\end{equation}

we have

\begin{dmath}\label{eqn:modernOptics:ProblemSet2:P3:410}
\gamma_{12}
= \frac{\Gamma_{12}}{\sqrt{I_1^2}}
= \frac{\Gamma_{12}}{I_1}
=
2 \frac{J_1(\pi \theta_s d/\lambda)}{\pi \theta_s d/\lambda}.
\end{dmath}

To calculate the total intensity we have

\begin{dmath}\label{eqn:modernOptics:ProblemSet2:P3:430}
I
= 2 I_1 + 2 \Real \Gamma_{12}
= 2 I_1 ( 1 + \Real \gamma_{12} )
\end{dmath}

so that
\begin{subequations}
\begin{equation}\label{eqn:modernOptics:ProblemSet2:P3:450}
I_{\mathrm{max}} = 2 I_1 ( 1 + \Abs{\gamma_{12}} )
\end{equation}
\begin{equation}\label{eqn:modernOptics:ProblemSet2:P3:470}
I_{\mathrm{min}} = 2 I_1 ( 1 - \Abs{\gamma_{12}} )
\end{equation}
\end{subequations}

So that for the visibility

\begin{dmath}\label{eqn:modernOptics:ProblemSet2:P3:490}
\calV = \frac
{
I_{\mathrm{max}}
- I_{\mathrm{max}}
}
{
I_{\mathrm{max}}
+ I_{\mathrm{max}}
}
=
\frac{2 I_1 2 \Abs{\gamma_{12}}}{4 I_1}
= \Abs{\gamma_{12}}
=
2 \Abs{\frac{J_1(\pi \theta_s d/\lambda)}{\pi \theta_s d/\lambda}}.
\end{dmath}

The Bessel function ratio that we have in the absolute values here is plotted in \cref{fig:modernOpticsProblemSet2Problem3:modernOpticsProblemSet2Problem3Fig3}.

\imageFigure{../../figures/phy485/modernOpticsProblemSet2Problem3Fig3}{Sinc like first order Bessel function}{fig:modernOpticsProblemSet2Problem3:modernOpticsProblemSet2Problem3Fig3}{0.3}

We find numerically that \(J_1(x)/x = 0.25\) occurs for \(x = 2.22\), so 50 \% visibility for the 500 nm average wavelength at 0.5 degrees occurs when

\begin{dmath}\label{eqn:modernOptics:ProblemSet2:P3:510}
d
= \frac{2.22 \lambda}{\pi \theta_s}
= \frac{2.22 \times 500 \times 10^{-9} \text{m}}{\pi \frac{\pi}{180} \inv{2}}
= 0.04 \text{mm}
\end{dmath}

% J_1 plot, unused:
%For The behaviour of this Bessel function as plotted in \cref{fig:modernOpticsProblemSet2Problem3:modernOpticsProblemSet2Problem3Fig2} will drive the visibility
%\imageFigure{../../figures/phy485/modernOpticsProblemSet2Problem3Fig2}{Bessel \(J_1\)}{fig:modernOpticsProblemSet2Problem3:modernOpticsProblemSet2Problem3Fig2}{0.3}

\paragraph{Part \ref{modernOptics:problemSet2:3b}.  }

%Borrowing \cref{fig:Black_body:wikipediaPlanksBlack_body} from \citep{wiki:planks}, we see that around the 500 nm range we can approximate the 5000 K spectral distribution as a Gaussian near the peak as in
%
%\imageFigure{../../figures/phy485/wikipediaPlanksBlack_body}{Plank's law distributions}{fig:Black_body:wikipediaPlanksBlack_body}{0.3}

\fxwarning{rework adding numerical evaluation of finite coherence time}{Unsuprisingly, I lost marks on this part of my solution.  A numeric result was desired, but I think I should have approached this much differently.}

Suppose that we have two sources of identical amplitude, each separated from the average frequency by half the spectral width

\begin{subequations}
\begin{dmath}\label{eqn:modernOptics:ProblemSet2:P3:530}
\Psi_1 = \sqrt{I_0} e^{i (\overbar{\omega} - \Delta \omega) t}
\end{dmath}
\begin{dmath}\label{eqn:modernOptics:ProblemSet2:P3:550}
\Psi_2 = \sqrt{I_0} e^{i (\overbar{\omega} + \Delta \omega) t}.
\end{dmath}
\end{subequations}

To compute the correlation of these we compute

\begin{dmath}\label{eqn:modernOptics:ProblemSet2:P3:570}
\Psi_1(t) \Psi_2^\conj(t + \tau)
=
I_0 e^{
i (\overbar{\omega} - \Delta \omega) t
-i (\overbar{\omega} + \Delta \omega) (t + \tau)
}
=
I_0 e^{ -2 i \Delta \omega t - i \overbar{\omega} \tau},
\end{dmath}

so that

\begin{dmath}\label{eqn:modernOptics:ProblemSet2:P3:590}
\gamma
= \expectation{ e^{ -2 i \Delta \omega t - i \overbar{\omega} \tau} }
= e^{- i \overbar{\omega} \tau }
\expectation{ e^{ -2 i \Delta \omega t } }
= e^{- i \overbar{\omega} \tau }
\inv{2 \tau_a}
\int_{-\tau_a}^{\tau_a} e^{ -2 i \Delta \omega t } dt
=
e^{- i \overbar{\omega} \tau }
\inv{2 \tau_a}
\frac{\sin(2 \Delta \omega \tau_a)}{\Delta \omega }
=
e^{- i \overbar{\omega} \tau } \sinc( 2 \Delta \omega \tau_a )
\end{dmath}

Our intensity is

\begin{dmath}\label{eqn:modernOptics:ProblemSet2:P3:610}
I
= 2 I_0 + 2 I_0 \Real \gamma
= 2 I_0 \left( 1 + \cos( \overbar{\omega} \tau ) \right) \sinc( 2 \Delta \omega \tau_a )
\end{dmath}

As the spectral width increases for a fixed period of observation \(\tau_a\) our intensity dies off from its maximum to the average \(2 I_0\).  This is sketched roughly in \cref{fig:modernOpticsProblemSet2Problem3:modernOpticsProblemSet2Problem3Fig4}

\imageFigure{../../figures/phy485/modernOpticsProblemSet2Problem3Fig4}{Possible fringes from two sources at the boundaries of the spectral width}{fig:modernOpticsProblemSet2Problem3:modernOpticsProblemSet2Problem3Fig4}{0.2}

%FIXME: why a dependence on the observation time \(\tau_a\)?  Sinc will flatten with \(\tau_a \rightarrow \infty\).

\paragraph{Part \ref{modernOptics:problemSet2:3c}.  }

\fxwarning{rework, also need numeric result for this part of the problem}{Also lost marks for this response.  Numeric result was desired (which I knew, but was rushing by this point)}

With a coherence time inversely proportional to the frequency

\begin{dmath}\label{eqn:modernOptics:ProblemSet2:P3:630}
t_c \sim \inv{\Delta \omega} = \frac{\Hbar}{\kB T},
\end{dmath}

a larger spectral width (larger T) will result in a smaller coherence time, and a requirement for faster detection circuitry.  This is clearly more of an issue than the spatial coherence when the distance to the object is very far.  One such example is the stellar interferometer discussed in \citep{hecht1998hecht} \S 12.4.2 where this coherence time was used to indirectly determine the diameter of a stellar source.

% didn't work for P1.  don't use here either.
%\makesubanswer{TODO.}{modernOptics:problemSet2:3a}
%\makesubanswer{TODO.}{modernOptics:problemSet2:3b}
%\makesubanswer{TODO.}{modernOptics:problemSet2:3c}
} % makeanswer
