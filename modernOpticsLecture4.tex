%
% Copyright � 2012 Peeter Joot.  All Rights Reserved.
% Licenced as described in the file LICENSE under the root directory of this GIT repository.
%
%\input{../blogpost.tex}
%\renewcommand{\basename}{modernOpticsLecture4}
%\renewcommand{\dirname}{notes/phy485/}
%\newcommand{\keywords}{Optics, PHY485H1F, diffraction, diffraction integral, Huygens-Fresnel principle}
%\input{../peeter_prologue_print2.tex}
%\beginArtNoToc
%\generatetitle{PHY485H1F Modern Optics.  Lecture 4: Diffraction.  Taught by Prof.\ Joseph Thywissen}
\label{chap:modernOpticsLecture4}

%\section{Disclaimer}
%
%Peeter's lecture notes from class.  May not be entirely coherent.
%
\section{Context}

We start the class with a green laser setup, where the light is displayed on the screen, then also after going through a single and double slit, as illustrated in \cref{fig:modernOpticsLecture4:modernOpticsLecture4Fig1}.

\imageFigure{../../figures/phy485/modernOpticsLecture4Fig1}{Laser on screen}{fig:modernOpticsLecture4:modernOpticsLecture4Fig1}{0.4}

We see also that a blue laser diffracts less.  The bigger the wavelength, the harder it is to ignore.  We can consider this a breakdown of geometric optics.

\section{Diffraction}
\index{diffraction}

We'll want to consider systems of this sort (light source, object in between, goes some distance, then observed) mathematically.  We consider the geometry of \cref{fig:modernOpticsLecture4:modernOpticsLecture4Fig2} where \(\BR = \Br - \Br'\), and \(\BR_s = \Br_s - \Br'\), and \(R = \Abs{\BR}\), \(R_s = \Abs{\BR_s}\).

\imageFigure{../../figures/phy485/modernOpticsLecture4Fig2}{Diffracting object (i.e. aperture)}{fig:modernOpticsLecture4:modernOpticsLecture4Fig2}{0.4}

We have two approximations to the full problem

\begin{enumerate}
\item A scalar theory can suffice.
\item The region of interest (and source) are paraxial.
\end{enumerate}

Why a scalar theory?  If we have a plane wave polarization

\begin{dmath}\label{eqn:modernOpticsLecture4:10}
\BE(\Br, t) = \left( E_1 \xcap + E_2 \ycap \right) e^{ i \Bk \cdot \Br - i \omega t}
\end{dmath}

With the principle of superposition

\begin{enumerate}
\item Solve for the \(x\) polarization.
\item Solve for the \(y\) polarization.
\item Vector addition of result.
\end{enumerate}

We will assume no mixing, so that we can treat just one component.

Reading: \S 8.3.1 \citep{born1980principles}, \S 9.8 \citep{jackson1975cew}.  The first goes and proves that the scalar theory is sufficient under this conditions.

We'd like to solve the wave equation with these approximations.

\begin{dmath}\label{eqn:modernOpticsLecture4:30}
\spacegrad^2 E = \inv{c^2} \PDSq{t}{E}
\end{dmath}

We will use a monochromatic wave so that we can write the electric field magnitude as a vector function times a time phase term

\begin{dmath}\label{eqn:modernOpticsLecture4:50}
E = \Psi(\Br) e^{-i \omega t}
\end{dmath}

We find

\boxedEquation{eqn:modernOpticsLecture4:70}{
\left( \spacegrad^2 + \Bk^2 \right) \Psi(\Br) = 0
}

This is called the Helmholtz equation.

It turns out that the solution to this equation is generally written out as the surface integral

% latex: \oiint gives not found.  What package was that from?
\begin{dmath}\label{eqn:modernOpticsLecture4:90}
\Psi(\Br) = \iint da' \left( \Psi(\Br') \spacegrad' G - G \spacegrad' \Psi(\Br') \right) \cdot \ncap
\end{dmath}

Here \(\ncap\) is the unit normal perpendicular to the surface, and the Green function of the Helmholtz equation is

\begin{dmath}\label{eqn:modernOpticsLecture4:110}
G(\Br, \Br') = -\frac{e^{i k R}}{4 \pi R} = -\frac{e^{i k \Abs{\Br - \Br'}}}{4 \pi \Abs{\Br - \Br'}},
\end{dmath}

%where \(R = \Abs{\Br - \Br'}\).
It is somewhat messy, but relatively straightforward to demonstrate \citep{phy456:helmoltzGreens} that this Green's function works to solve the forced Helmoltz equation

\begin{equation}\label{eqn:modernOpticsLecture4:helmholtzGreens:10}
\left( \spacegrad^2 + \Bk^2\right) \Psi_\Bk(\Br) = s(\Br),
\end{equation}

with solution

\begin{equation}\label{eqn:modernOpticsLecture4:helmholtzGreens:50}
\Psi_\Bk(\Br) = \int G(\Abs{\Br - \Br'}) s(\Br') d^3 \Br'.
\end{equation}

However, it is far from obvious how to apply this to the homogeneous Helmoltz equation.  The tricks involved (application of Green's theorem to a spherical volume with the center deleted) can be found in \S A.2, \S 10.4 of \citep{hecht1998hecht}, and also in \S 5.2 of \citep{fowles1989introduction}.  The end result of that trickery is called the Kirchhoff Integral Theorem.

Is this Green's function reasonable seeming?  As illustrated in \cref{fig:modernOpticsLecture4:modernOpticsLecture4Fig3}) this isn't an entirely unsurprising seeming Green's function for this problem.  We have the \(e^{i k R}\) type of phase factor that we expected (and guessed in the geometric optics treatment, and also have the \(1/R\) factor that we need to retain power at a distance \(R\).

\imageFigure{../../figures/phy485/modernOpticsLecture4Fig3}{Wave function at the aperture}{fig:modernOpticsLecture4:modernOpticsLecture4Fig3}{0.2}

Also note that the primed gradient is taken with respect to the coordinates of \(\Br'\)

\begin{dmath}\label{eqn:modernOpticsLecture4:130}
\spacegrad' = \Be_m \PD{{x'}_m}{}
\end{dmath}

If we take the gradient of the Green's function we find

\begin{dmath}\label{eqn:modernOpticsLecture4:150}
\spacegrad \left(
\frac{ e^{i k r} }{r}
\right) = \rcap \left( i k - \inv{r} \right)
\frac{e^{i k r}}{r}
\end{dmath}

Applying this to our problem we find

\boxedEquation{eqn:modernOpticsLecture4:170}{
\Psi(\Br) = -\inv{4 \pi} \iint \frac{e^{i k R}}{R}
\ncap \cdot
\left(
\spacegrad' \Psi(\Br') + \left( i k - \inv{R} \right) \frac{\BR}{R} \Psi(\Br') \right)
da'.
}

Here \(\BR = \Br - \Br'\) and \(da' = dx' dy'\) or \(\rho' d\rho' d\theta'\).  We are going to neglect the surface at \(\infty\) as illustrated in \cref{fig:modernOpticsLecture4:modernOpticsLecture4Fig4}.

\imageFigure{../../figures/phy485/modernOpticsLecture4Fig4}{Neglecting the surface at infinity}{fig:modernOpticsLecture4:modernOpticsLecture4Fig4}{0.4}

This neglect is justified for example in Jackson, cited above.

\section{A calculated example: pinhole}
\index{pinhole}

With placement of our origin at the pinhole, so that \(\Br' = 0\), \(\BR = \Br\), \(\BR_s = \Br_s\), we want to consider the geometry of \cref{fig:modernOpticsLecture4:modernOpticsLecture4Fig5}.

\imageFigure{../../figures/phy485/modernOpticsLecture4Fig5}{Source, aperture and observation point}{fig:modernOpticsLecture4:modernOpticsLecture4Fig5}{0.4}

Our spherical wave function at the aperture is

\begin{dmath}\label{eqn:modernOpticsLecture4:190}
\Psi(\Br') = A \frac{e^{i k R_s}}{R_s}
\end{dmath}

so that

\begin{dmath}\label{eqn:modernOpticsLecture4:210}
\spacegrad' \Psi
= - A \frac{\Br_s - \Br'}{\Abs{\Br_s - \Br'}}
\left( i k - \inv{R_s} \right) \frac{e^{i k R_s}}{R_s}
= - A \frac{\Br_s}{\Abs{\Br_s}}
\left( i k - \inv{R_s} \right) \frac{e^{i k R_s}}{R_s}
= A \ncap \left( i k - \inv{R_s} \right) \frac{e^{i k R_s}}{R_s}.
\end{dmath}

Our resulting wave function is then

\begin{dmath}\label{eqn:modernOpticsLecture4:230}
\Psi(\Br) = - \frac{A}{4 \pi} \iint da' \frac{e^{i k ( R + r_s)}}{R r_s}
\left(
\ncap \cdot \ncap
\left( i k - \inv{r_s}
\right) + \ncap \cdot \rcap
\left( i k - \inv{r}
\right)
\right)
\end{dmath}

Now, in all these \(ik - 1/r_s\) we have \(k\) of order \(1/\lambda\) and \(1/r_s\) is of order \(1/r\) or \(1/r_s\).

Recall from geometric optics that we used

\begin{dmath}\label{eqn:modernOpticsLecture4:250}
\spacegrad \left( \BE_0 e^{i\phi(\Br)} \right) \approx i (\BE_0 \cdot \spacegrad \phi ) e^{i \phi(\Br)},
\end{dmath}

With an assumption

\begin{dmath}\label{eqn:modernOpticsLecture4:270}
\lambda \ll r, r_s,
\end{dmath}

and

\begin{dmath}\label{eqn:modernOpticsLecture4:290}
\lambda \ll d \ll r, r_s,
\end{dmath}

where \(d\) is the ``typical object size'', so that we have

\begin{dmath}\label{eqn:modernOpticsLecture4:310}
\Psi(\Br) = - \frac{A}{4 \pi} \iint da' \frac{e^{i k ( R + r_s)}}{R r_s} \left(
\ncap \cdot \ncap \left( i k - \cancel{\inv{r_s}} \right) + \ncap \cdot \rcap
\left( i k - \cancel{\inv{r}}
\right)
\right)
\end{dmath}

or with \(\theta\) as illustrated,

\boxedEquation{eqn:modernOpticsLecture4:330}{
\Psi(\Br) = \frac{A}{\lambda i} \iint da' \frac{e^{i k ( r + r_s)}}{r r_s} k(\theta),
}

where

\begin{dmath}\label{eqn:modernOpticsLecture4:350}
k(\theta) = \inv{2} \left( 1 + \cos \theta \right),
\end{dmath}

is the ``obliquity factor''.

This is called the \underlineAndIndex{Huygens-Fresnel} principle.

% This is messed up.  Looks like he was trying to give us a glance at what was to come.
%\section{Generalizing}
%
%Should we open up the pinhole, we have to adjust things accordingly, and find
%
%\begin{dmath}\label{eqn:modernOpticsLecture4:370}
%\Psi(\Br) = \frac{k A}{2 \pi i} \iint \frac{e^{i k (R + R_s)}}{ r r_s } k(\theta) d^2 \Br'
%\end{dmath}
%
%This is called the \underlineAndIndex{diffraction integral}.

%\EndArticle
