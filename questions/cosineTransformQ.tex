%
% Copyright � 2012 Peeter Joot.  All Rights Reserved.
% Licenced as described in the file LICENSE under the root directory of this GIT repository.
%
% pick one:
%\input{../latex/assignment.tex}
\input{../latex/blogpost.tex}
\renewcommand{\basename}{cosineTransformQ}
\renewcommand{\dirname}{notes/phy485/}
%\newcommand{\dateintitle}{}
%\newcommand{\keywords}{}

\input{../latex/peeter_prologue_print2.tex}

\beginArtNoToc

If we want to compute the FT of a time shifted function with a normal FT we have

% t - t_0 = u
\begin{dmath}\label{eqn:cosineTransformQ:10}
\int_{-\infty}^\infty
f(t - t_0) e^{-i\omega t} dt
=
\int_{-\infty}^\infty
f(u) e^{-i\omega (u + t_0)} du
= \tilde{f}(\omega) e^{-i \omega t_0},
\end{dmath}

or for a frequency shift

% omega - omega_0 = omega'
\begin{dmath}\label{eqn:cosineTransformQ:30}
\int_{-\infty}^\infty
\tilde{f}(\omega - \omega_0) e^{i\omega t} d\omega
=
\int_{-\infty}^\infty
\tilde{f}(\omega') e^{i(\omega' + \omega_0) t} d\omega'
=
e^{i \omega_0 t} f(t)
\end{dmath}

With a cosine transform, this doesn't work out so nicely, because we have to deal with the impact of the change of variables at the origin.  Example, for even \(f(\tau)\), with cosine transform \(\tilde{f}(\omega)\), we have

% t - t_0 = t' ; t = t' + t_0
\begin{dmath}\label{eqn:cosineTransformQ:50}
\int_0^\infty f(\tau - \tau_0) \cos( \omega \tau ) d\tau
=
\int_{-t_0}^\infty f(\tau') \cos( \omega (\tau' + \tau_0) ) d\tau'
=
\int_{-t_0}^0 f(\tau - \tau_0) \cos( \omega \tau ) d\tau
+
\int_{0}^\infty f(\tau')
\left(
\cos (\omega \tau') \cos(\omega \tau_0)
-\sin (\omega \tau') \sin(\omega \tau_0)
\right)
d\tau'
=
\int_{0}^{\tau_0} f(\tau - \tau_0) \cos( \omega \tau ) d\tau
+ \cos(\omega \tau_0) \tilde{f}(\omega)
-\sin(\omega \tau_0)
\int_0^\infty f(\tau') \sin (\omega \tau')
\end{dmath}

Like a normal FT we have the multiplicative factor (the \(\cos(\omega\tau_0)\)), but we now also have a pesky additional integral around the origin, because we didn't have the convienent \([-\infty, \infty]\) that absorbs any impact of change of variables in the shift, and also have a sine transform term showing up (if we were integrating over a symmetric interval this product of even and odd functions would integrate out, but that doesn't die so nicely for the \([0, \infty]\) interval.

Is there a way to deal with time and frequency shifts in cosine transforms that's comparable to the results for normal Fourier transforms?

%\EndArticle
\EndNoBibArticle
