%
% Copyright � 2012 Peeter Joot.  All Rights Reserved.
% Licenced as described in the file LICENSE under the root directory of this GIT repository.
%
% pick one:
%\input{../assignment.tex}
%\input{../blogpost.tex}
%\renewcommand{\basename}{paraxialWaveApproximateSolution}
%\renewcommand{\dirname}{notes/phy485/}
%%\newcommand{\dateintitle}{}
%%\newcommand{\keywords}{}
%
%\input{../peeter_prologue_print2.tex}
%
%\beginArtNoToc
%
%\generatetitle{Fresnel form approximate solution to paraxial wave equation}
%%\chapter{Fresnel form approximate solution to paraxial wave equation}
\index{paraxial wave equation}

\makeproblem{Apply the paraxial wave equation operator to a Fresnel approximate form}{pr:paraxialWaveApproximateSolution:1}{

In some supplementary class notes, it is stated that

\begin{dmath}\label{eqn:paraxialWaveApproximateSolution:1}
h(x, y, z) = \inv{z} e^{i k z} e^{i k (x^2 + y^2)/2z }
\end{dmath}

is an exact solution to the \underlineAndIndex{paraxial wave equation}

\begin{dmath}\label{eqn:paraxialWaveApproximateSolution:10}
\spacegrad_{\txtT}^2 u + 2 i k \PD{z}{u} = 0.
\end{dmath}

From our lectures, this doesn't seem possible, since we found that this Fresnel like function was an approximation to the \(u_{00}\) function for large \(z\).  Calculate this directly and verify this suspicion.

} % makeproblem

\makeanswer{pr:paraxialWaveApproximateSolution:1}{

Let's first apply the \(\partial_{xx}\) portion of the transverse Laplacian.  We find

\begin{dmath}\label{eqn:paraxialWaveApproximateSolution:30}
\PDSq{x}{h}
=
\PD{x}{} \PD{x}{} \left(
\inv{z} e^{i k z} e^{ i k (x^2 + y^2)/2z }
\right)
=
\inv{z} e^{i k z}
e^{ i k y^2 /2z }
\PD{x}{} \PD{x}{} \left(
e^{ i k x^2 /2z }
\right)
=
\inv{z} e^{i k z}
e^{ i k y^2 /2z }
\PD{x}{} \left(
\frac{i k x}{z} e^{ i k x^2 /2z }
\right)
=
\inv{z} e^{i k z}
e^{ i k y^2 /2z }
\left(
\frac{i k }{z}
-\frac{k^2 x^2}{z^2}
\right)
e^{ i k x^2 /2z }
=
\left(
\frac{i k }{z}
-\frac{k^2 x^2}{z^2}
\right)
h
\end{dmath}

This gives us, for \(r^2 = x^2 + y^2\)

\begin{dmath}\label{eqn:paraxialWaveApproximateSolution:50}
\spacegrad_{\txtT}^2 h =
\left(
\frac{2 i k }{z}
-\frac{k^2 r^2}{z^2}
\right)
h.
\end{dmath}

For the first partial with respect to \(z\) we find

\begin{dmath}\label{eqn:paraxialWaveApproximateSolution:70}
\PD{z}{h} =
-\inv{z^2} e^{i k z} e^{ i k r^2 /2z }
+
\inv{z}
\left(
i k - \frac{i k r^2}{2 z^2}
\right)
e^{i k z} e^{ i k r^2 /2z }
=
\left(
- \inv{z} +
i k - \frac{i k r^2}{2 z^2}
\right) h
\end{dmath}

Putting things together we have

\begin{dmath}\label{eqn:paraxialWaveApproximateSolution:90}
\left( \spacegrad_{\txtT}^2 + 2 i k \PD{z}{} \right) h
=
\left(
\cancel{
\frac{2 i k }{z}
}
-
\cancel{\frac{k^2 r^2}{z^2} }
+
2 i k \left(
\cancel{
- \inv{z}
} +
i k -
\cancel{
\frac{i k r^2}{2 z^2}
}
\right)
\right)
h
=
- 2 k^2 h \ne 0
\end{dmath}

However, since \(h \rightarrow 0\) as \(z \rightarrow \infty\), this does at least give zero in the far \(z\) limit.
} % makeanswer

%\EndArticle
%\EndNoBibArticle
